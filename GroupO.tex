\documentclass[12pt]{article}

\usepackage{graphicx}
\begin{document}

	
	
	\title{Automation of the Manual System for Government Students’ Hall Registration at Makerere University.}
	
	\author{Group 0}
	
	\maketitle
	\section{Introduction}
	\subsection{Background of the study}
	\paragraph{}
	Student Hall Registration is the process where by every student has to register at his/her hall of attachment at the beginning of the first year after their admission at Makerere University.
	However for the govt students and the private-residents have to register every academic year for their respective programs,for instance govt-residents have to register for the change of the rooms and the non-residents have to also register for their accounts updates(for example if the student wants to change the mobile money number where he/she receives the upkeep from).
	The current system involves a lot of paper work where every student has to collect forms from his/her hall of residence which is time consuming and  doesn't favor the evening students who normally report at 5:00pm yet hall offices operate from 8:00am  to 4:30pm. 
	\subsection{Problem Statement}
		\paragraph{}
		The importance of developing an online government students hall registration has long been recognized by Makerere University. However, numerous registration processes are still plagued by delays and cost overruns, which
can frequently lead to delays and disruption of the study schedule for the students. Instances where government students are required to queue at their respective halls of attachment or residence every semester for long hours in the names of registering or validating their details with the University are inevitable. Today’s registration systems are becoming more and more technically complex and logistically challenging, which exposes the University’s operations to even more complex constraints with regards to its academic visions. Second, the traditional way of registering government students which involves a lot of paper work which in most cases is accompanied with errors and confusion when it comes to storage most especially due to the large number of students being enrolled by the University every academic year. These methods have long been blamed for their limitations in the vertical progress of the academic mecca including inability to cope with non-time-related precedence constraints and difficulty to evaluate and organize their plans as per their wish In summary, there is a need to develop and online hall registration system that will enable students register at their own ease and also help the university operate as per its schedule with every student satisfied and with no worries as far as their info and details with the University are concerned.
	\subsection{Objectives}
		\paragraph{}
	The long term goal of the research is to develop an online hall registration system.Hall registration is defined herein as is the process where by every student has to register at his/her hall of attachment at the beginning of the first year after their admission. The objective of the current study is to come up with an online system that will help cab the problems encountered in the process of hall registration. Particularly, the study has the following objectives:
	\begin{enumerate}
		
    \itemTo develop an online system to help reduce on the paper work involved during registration.
     \itemTo reduce on the time spent by students in traveling to their halls of attachment to register.
      \itemTo develop an online system which will help reduce on the number of students queuing every single day at their halls in the names registration.
      \itemTo enable students access their hall details and the admin to do their routine work of monitoring the progress.
       \item To provide a platform that will enable students carry out registration at anytime of their convenience. 
       \itemTo gain more reliable way of record keeping, since records online records are less affected by uncertainty.
       \itemTo attain a simple way of addressing the issue of errors. Errors are easily corrected.
       
	\end{enumerate}
The result of this study will be valuable to the students as well
as the university and related stakeholders in better planning and management of the persons that are admitted at the campus.
	\subsection{Scope}
		\paragraph{}
	This online system is to be used to register only government sponsored students at MUK.
	This project will avail to the students a platform through which their information can be captured every academic year which will include, Name of the student, Student No, Registration No, Telephone No, Hall of residence/attachment, Program, Age, Sex, etc.
	The system also gives the students a provision to update or change their details at any given time.
	When a student submits their info through the portal, they are required to wait for an admin confirmation message sent to their emails and telephone number.
	
	\subsection{Significance}
		\paragraph{}
	 Because it will solve the initially manual and paper work used, an online registration system will help reduce costs and man power involved during the registration process.
	Students and the University Staff will have more time to interact and view their statuses as far as their details are concerned.
		\maketitle
	\section{Literature Review}
		\paragraph{}
	A preliminary literature review shows that past studies are primarily
	focused on understanding and modeling the system that was in place a particular type of constraint, such as technological, \cite{DUMMY:1}contractual, resource, spatial, and information constraints. Limited progress has been made on classifying various constraints according \cite{ARTICLE:1}to their characteristics in a comprehensive manner. In terms of modeling and resolving constraints, various approaches have been recommended. For example, many CPM-based( Initial-is of characters per minute)\cite{BOOK:1} methods are applied to deal with time-related constraints; knowledge-based systems were used to automate work plan for the University; network-based optimization algorithms were developed to resolve constraints; and databases and visualization techniques, such as 3D, 4D,\cite{WEBSITE:1} and Virtual Reality (VR), are used to communicate and visualize constraints. What is missing from the past studies is a comprehensive and structured approach in managing constraints as far as the problem is concerned.
		\maketitle
	\section{Methodology}
		\paragraph{}
		The following objectives are going to be obtained with the help of the interfaces within this section
		\begin{enumerate}
	
		\itemTo develop an online system to help reduce on the paper work involved during registration.
		which will help reduce on the number of students queuing every single day at their halls in the names registration.
		The system allows a user to create an account and the students information is saved into the database that is later synchronized to the university database for verification.
		The verification code is sent to the user phone number used in registration and the email account of the user just in-case of any thing..
		\begin{figure}
			\includegraphics[width=\linewidth]{login.png}
			\caption{Login Interface}
			\label{fig:login Interface}
		\end{figure}
		\itemTo reduce on the time spent by students in traveling to their halls of attachment to register.
		\itemTo develop an online system which will help reduce on the number of students queuing every single day at their halls in the names registration.
		\itemTo enable students access their hall details and the admin to do their routine work of monitoring the progress.
		\item To provide a platform that will enable students carry out registration at anytime of their convenience. 
			\begin{figure}
			\includegraphics[width=\linewidth]{aboutus.png}
			\caption{aboutus Interface}
			\label{fig:AboutUs Interface}
		\end{figure}
		\itemTo gain more reliable way of record keeping, since records online records are less affected by uncertainty.
		\itemTo attain a simple way of addressing the issue of errors. Errors are easily corrected.
			\begin{figure}
			\includegraphics[width=\linewidth]{register.png}
			\caption{Register Interface}
			\label{fig:Register Interface}
		\end{figure}
	
	\begin{figure}
		\includegraphics[width=\linewidth]{form.png}
		\caption{form Interface}
		\label{fig:form Interface}
	\end{figure}


		\begin{figure}
		\includegraphics[width=\linewidth]{info.png}
		\caption{info Interface}
		\label{fig:info Interface}
	\end{figure}	
	\end{enumerate}

\newpage
	\bibliography{group} 
\bibliographystyle{ieeetr}
		
\end{document}